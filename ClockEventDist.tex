\documentclass{beamer}

\usepackage{centernot}
% \usepackage{beamerthemesplit} // Activate for custom appearance

\title{Time, Clocks, and the Ordering of Events in a Distributed System}
\subtitle{Leslie Lamport}
\author{Yuncong Zhang}
\date{April 23, 2020}

\begin{document}

\frame{\titlepage}

\section[Outline]{}
\frame{\frametitle{Outline}\tableofcontents}

\section{Total Ordering of Events}

\frame
{
  \frametitle{Events in Distributed System}
	\begin{columns}
	\begin{column}{0.6\textwidth}

		What does it mean by ``$a$ happens before $b$'' in distributed system?

		\begin{itemize}
			\item<2-> In one process, \textbf{earlier} events happens before \textbf{later} events
			\item<3-> \textbf{Sending message} happens before \textbf{receiving message}
			\item<4-> If $a$ happens before $b$, and $b$ happens before $c$, then $a$ happens before $c$
		\end{itemize}

	\end{column}
	\begin{column}{0.4\textwidth}

		\begin{figure}[ht!]
		\includegraphics[width=\textwidth]{files/events-messages.png}
		\caption{Events and Messages in Processes}
		\end{figure}


	\end{column}
	\end{columns}
}

\frame
{
  \frametitle{Partial Ordering of Events}
	\begin{columns}
	\begin{column}{0.6\textwidth}

		Denote by $a\to b$ if $a$ happens before $b$.

		\begin{itemize}
			\item<2-> ``$\to$'' defines a partial order.
			\item<3-> If $a\nrightarrow b$ and $b\nrightarrow a$ then $a$ and $b$ are \emph{concurrent}
			\item<4-> $a\to b$ is equivalent to saying one can go from $a$ to $b$ in the diagram by moving forward in time along process and message lines.
		\end{itemize}


	\end{column}
	\begin{column}{0.4\textwidth}

		\begin{figure}[ht!]
		\includegraphics[width=\textwidth]{files/events-messages.png}
		\caption{Events and Messages in Processes}
		\end{figure}


	\end{column}
	\end{columns}

}

\frame
{
  \frametitle{Partial Ordering of Events}
	\begin{columns}
	\begin{column}{0.6\textwidth}

		Example: $p_1\to r_4$
		\begin{itemize}
			\item<2-> $p_1\to q_2$
			\item<3-> $q_2\to q_4$
			\item<4-> $q_4\to r_3$
			\item<5-> $r_3\to r_4$
		\end{itemize}


	\end{column}
	\begin{column}{0.4\textwidth}

		\begin{figure}[ht!]
		\includegraphics[width=\textwidth]{files/events-messages.png}
		\caption{Events and Messages in Processes}
		\end{figure}


	\end{column}
	\end{columns}

}

\frame
{
	\frametitle{Logical Clocks}

	\begin{columns}
	\begin{column}{0.6\textwidth}

		Logical clock is an assignment of numbers on events
		\begin{itemize}
			\item<2-> Clock $C_i$ assigns number $C_i\langle a\rangle$ to event $a$ in process $P_i$
			\item<3-> Clock $C$ for the entire system defined by $C\langle a\rangle=C_i\langle a\rangle$ if $a$ is in process $P_i$
		\end{itemize}


	\end{column}
	\begin{column}{0.4\textwidth}

		\begin{figure}[ht!]
		\includegraphics[width=\textwidth]{files/ClockDist-Logical-Clock.png}
		\caption{Logical Clock}
		\end{figure}


	\end{column}
	\end{columns}


}

\frame
{
	\frametitle{Logical Clocks}

	\begin{columns}
	\begin{column}{0.6\textwidth}

		\begin{block}{Clock Condition}
			For any events $a,b$:

			if $a\to b$ then $C\langle a\rangle < C\langle b\rangle$
		\end{block}

		\begin{block}{Remark}
			The converse is not true:

			$C\langle a\rangle < C\langle b\rangle$ does not imply $a\to b$
		\end{block}


	\end{column}
	\begin{column}{0.4\textwidth}

		\begin{figure}[ht!]
		\includegraphics[width=\textwidth]{files/ClockDist-Logical-Clock.png}
		\caption{Logical Clock}
		\end{figure}


	\end{column}
	\end{columns}


}

\frame
{
	\frametitle{Logical Clocks}

	\begin{columns}
	\begin{column}{0.6\textwidth}

		Implement the logical clock:

		\begin{itemize}
			\item<2-> Each process $P_i$ increments $C_i$ between successive events
			\item<3-> If event $a$ is sending message $m$ from $P_i$, then $m$ contains timestamp $T_m=C_i\langle a\rangle$
			\item<4-> On receiving message $m$, $P_j$ sets $C_j$ to be greater than both $T_m$ and previous event
		\end{itemize}


	\end{column}
	\begin{column}{0.4\textwidth}

		\begin{figure}[ht!]
		\includegraphics[width=\textwidth]{files/ClockDist-Logical-Clock.png}
		\caption{Logical Clock}
		\end{figure}


	\end{column}
	\end{columns}


}

\frame
{
	\frametitle{Logical Clocks}

	\begin{columns}
	\begin{column}{0.6\textwidth}

		Process $Q$ receives message $p_1$, updates clock to $2$


	\end{column}
	\begin{column}{0.4\textwidth}

		\begin{figure}[ht!]
		\includegraphics[width=\textwidth]{files/ClockDist-Impl-Logical-Clock-1.png}
		\caption{Logical Clock}
		\end{figure}


	\end{column}
	\end{columns}


}

\frame
{
	\frametitle{Logical Clocks}

	\begin{columns}
	\begin{column}{0.6\textwidth}

		Process $P$ receives message $q_1$, updates clock to $2$


	\end{column}
	\begin{column}{0.4\textwidth}

		\begin{figure}[ht!]
		\includegraphics[width=\textwidth]{files/ClockDist-Impl-Logical-Clock-2.png}
		\caption{Logical Clock}
		\end{figure}


	\end{column}
	\end{columns}


}

\frame
{
	\frametitle{Logical Clocks}

	\begin{columns}
	\begin{column}{0.6\textwidth}

		Proceeds until $Q$ sends a message to $R$ at event $q_4$ with timestamp 4


	\end{column}
	\begin{column}{0.4\textwidth}

		\begin{figure}[ht!]
		\includegraphics[width=\textwidth]{files/ClockDist-Impl-Logical-Clock-3.png}
		\caption{Logical Clock}
		\end{figure}


	\end{column}
	\end{columns}


}

\frame
{
	\frametitle{Logical Clocks}

	\begin{columns}
	\begin{column}{0.6\textwidth}

		Process $R$ receives the message with timestamp 4, and updates clock to 5


	\end{column}
	\begin{column}{0.4\textwidth}

		\begin{figure}[ht!]
		\includegraphics[width=\textwidth]{files/ClockDist-Impl-Logical-Clock-4.png}
		\caption{Logical Clock}
		\end{figure}


	\end{column}
	\end{columns}


}

\frame
{
	\frametitle{Logical Clocks}

	\begin{columns}
	\begin{column}{0.6\textwidth}

		Process $Q$ sends message to $P$ with timestamp 5


	\end{column}
	\begin{column}{0.4\textwidth}

		\begin{figure}[ht!]
		\includegraphics[width=\textwidth]{files/ClockDist-Impl-Logical-Clock-5.png}
		\caption{Logical Clock}
		\end{figure}


	\end{column}
	\end{columns}


}

\frame
{
	\frametitle{Logical Clocks}

	\begin{columns}
	\begin{column}{0.6\textwidth}

		Process $P$ updates clock on receiving message with timestamp 5.
		Clocks of processes $Q$ and $R$ are not affected by messages.


	\end{column}
	\begin{column}{0.4\textwidth}

		\begin{figure}[ht!]
		\includegraphics[width=\textwidth]{files/ClockDist-Impl-Logical-Clock-6.png}
		\caption{Logical Clock}
		\end{figure}


	\end{column}
	\end{columns}


}

\frame
{
	\frametitle{Total Ordering}
	\begin{columns}
	\begin{column}{0.6\textwidth}


		With the logical clock, we can now define a total order ``$\Rightarrow$'' for all events.

		\begin{itemize}
			\item<2-> Define a total order $\prec$ over the processes
			\item<3-> For events $a$ in $P_i$ and $b$ in $P_j$, $a\Rightarrow b$ if and only if either
				\begin{itemize}
					\item $C_i\langle a\rangle < C_j\langle b\rangle$ or;
					\item $C_i\langle a\rangle = C_j\langle b\rangle$ and $P_i\prec P_j$
				\end{itemize}
		\end{itemize}

	\end{column}
	\begin{column}{0.4\textwidth}

		\begin{figure}[ht!]
		\includegraphics[width=\textwidth]{files/ClockDist-Total-Order.png}
		\caption{Logical Clock}
		\end{figure}


	\end{column}
	\end{columns}
}

\section{Mutual Exclusion}
\frame
{
  \frametitle{Mutual Exclusion}
  Consider a system composed of a fixed collection of processes which share a single resource. Only one process can use the resource at a time.

  \begin{itemize}
  	\item (I) A process which has been granted the resource must release it before it can be granted to another process.
  	\item (II) Different requests for the resource must be granted in the order in which they are made.
  	\item (III) If every process which is granted the resource eventually releases it, then every request is eventually granted.
  \end{itemize}
}

\frame
{
  \frametitle{Mutual Exclusion}

  To simplify the implementation, some assumptions are needed to avoid extra details.
  \begin{itemize}
  	\item<2-> For any $P_i$ and $P_j$, the messages sent from $P_i$ to $P_j$ are received in order
  	\item<3-> Every message is eventually received
  	\item<4-> A process can send messages directly to every other process
  \end{itemize}
}

\frame
{
  \frametitle{Mutual Exclusion}

  Each process maintains its own \emph{request queue}.

  \begin{itemize}
  	\item<2-> If process $P_i$ wants the resource, it sends a message $\langle T_m:P_i\text{ \emph{requests resource}}\rangle$ to every other process, and puts the message on its request queue, where $T_m$ is the message timestamp
  	\item<3-> When process $P_j$ receives $\langle T_m:P_i\text{ \emph{requests resource}}\rangle$, it puts the message on its request queue, and sends an acknowledgement message to $P_i$
  \end{itemize}
}

\frame
{
  \frametitle{Mutual Exclusion}

  Releasing the resource works in similar way.

  \begin{itemize}
  	\item<2-> If process $P_i$ wants to release the resource, it sends a message $\langle T_m:P_i\text{ \emph{releases resource}}\rangle$ to every other process, and removes any $\langle T_m:P_i\text{ \emph{requests resource}}\rangle$ from its request queue
  	\item<3-> When process $P_j$ receives $\langle T_m:P_i\text{ \emph{releases resource}}\rangle$, it removes any $\langle T_m:P_i\text{ \emph{requests resource}}\rangle$ from its request queue
  \end{itemize}
}

\frame
{
  \frametitle{Mutual Exclusion}

  $P_i$ is granted the resource if the following conditions hold.

  \begin{itemize}
  	\item<2-> There is a $\langle T_m:P_i\text{ \emph{requests resource}}\rangle$ in its request queue ordered before any other messages inside the queue by relation ``$\Rightarrow$''
  	\item<3-> $P_i$ received the acknowledgements from every other process
  \end{itemize}
}

\frame
{
  \frametitle{Mutual Exclusion}
  The above protocol satisfies conditions (I)(II) and (III).

  \begin{itemize}
  	\item<2-> \textbf{Conditions I and II:} When $\langle T_m:P_i\text{ \emph{requests resource}}\rangle$ is sent, no request ordered after this request will be granted before this request is released, because they either:
  		\begin{itemize}
  			\item Are not acknowledged by $P_i$
  			\item Received $\langle T_m:P_i\text{ \emph{requests resource}}\rangle$ which is ordered before them in their queues
  		\end{itemize}
		\item<3-> \textbf{Condition III:} If all the requests before $\langle T_m:P_i\text{ \emph{requests resource}}\rangle$ are released, this request will be ordered before any other requests in $P_i$'s queue
  \end{itemize}
}

\section{Physical Clock}
\frame
{
  \frametitle{Physical Clock}
  Let $t$ denote the physical time, and $C_i(t)$ be the value of clock $C_i$.

  \begin{itemize}
  	\item\pause $C_i(t)$ is continuous and differentiable function of $t$ except for jump discontinuities
  	\item\pause $dC_i(t)/dt$ is the rate of the clock at $t$, assumed that $\left|dC_i(t)/dt-1\right|<\kappa\ll 1$
  \end{itemize}

  \pause
  \begin{figure}[ht!]
  \includegraphics[width=0.4\textwidth]{files/ClockDist-Physical-Clock.png}
  \end{figure}

}

\frame
{
  \frametitle{Physical Clock}
  We need a protocol to make sure that all the $C_i(t)$'s are synced, i.e. for all $i,j$:

  \[|C_i(t)-C_j(t)|<\varepsilon\]

  \pause
  We need some assumptions and notations before describing the protocol.

  \begin{itemize}
  	\item\pause If a message $m$ is sent at time $t$, received at time $t'$, the total delay is $\nu_m=t'-t$, which is not known to the receiver
  	\item\pause We assume that the receiving process knows some minimum delay $\mu_m\leq\nu_m$
  	\item\pause Denote by $\xi_m=\nu_m-\mu_m$ the \emph{unpredictable delay}
  \end{itemize}
}

\frame
{
  \frametitle{Physical Clock}
  The protocol executes as follows.

  \begin{itemize}
  	\item<2-> If $P_i$ does not receive message at time $t$, then $P_i$ does not modify $C_i$, and $dC_i(t)/dt\approx 1$
  	\item<3-> When $P_i$ sends a message $m$, it appends timestamp $T_m=C_i(t)$ in the message
  	\item<4-> If at time $t'$, $P_i$ receives a message $m$ with timestamp $T_m$, $P_i$ resets $C_i(t')$ to $\max(C_i(t'), T_m+\mu_m)$
  \end{itemize}
}

\frame
{
  \frametitle{Physical Clock}
  \begin{block}{Theorem}
  Assume a strongly connected graph of processes with diameter $d$ follows the above protocol. Assume that for any message $m$, $\mu_m\leq\mu$ for some constant $\mu$, and that for all $t\geq t_0$:
  \begin{enumerate}
  	\item $|dC_i(t)/dt-1|\leq\kappa\ll 1$
  	\item every $\tau$ seconds a message $m$ with $\xi_m<\xi$ is sent over every arc, where $\tau$ and $\xi$ are constants
  \end{enumerate}
  Then for all $i,j$: $|C_i(t)-C_j(t)|<\varepsilon$ where $\varepsilon\approx d(2\kappa\tau+\xi)$ for all $t\gtrsim t_0+\tau d$ and $\mu+\xi\ll\tau$
  \end{block}
}

\frame
{
  \frametitle{Physical Clock}
  \begin{block}{Proof Sketch}
  \begin{enumerate}
  	\item<2-> Prove that for any $i,j$ and any $t,t_1$ with $t_1\geq t_0$ and $t\geq t_1+d(\tau+\nu)$:
  	\[C_i(t)\geq C_j(t_1)+(1-\kappa)(t-t_1)-d\xi\]
  	\item<3-> Prove that for any $t,t_x$ with $t\geq t_x\geq t_0+\mu/(1-\kappa)$ there is a process $P_q$ and a time $t_1$ with $t_x-\mu/(1-\kappa)\leq t_1\leq t_x$ such that for all $i$:
  	\[C_i(t)\leq C_q(t_1)+(1+\kappa)(t-t_1)\]
  	\item<4-> Prove that for all $i,j$:
  	\[|C_i(t)-C_j(t)|\lesssim d(2\kappa\tau+\xi)\]
  	for all $t\gtrsim t_0+d\tau$
  \end{enumerate}
  \end{block}
}

\frame
{
  \frametitle{Physical Clock}
  \begin{block}{Step 1}
  \begin{enumerate}
  	\item<1-> Define $C_i^t$ to be a clock set equal to $C_i$ at time $t$ and runs at the same rate as $C_i$, but is never reset. Then we have for any $t'\geq t$, $C_i^t(t')\leq C_i(t')$
  	\item<2-> Suppose $P_1$ at time $t_1$ sends a message to $P_2$ which is received at $t_2$, then for any $t\geq t_2$
  	\[C_2^{t_2}(t)\geq C_1(t_1)+(1-\kappa)(t-t_1)-\xi\]
  	\item<3-> For any $P$ and $P'$, there is a sequence $P=P_0,P_1,\cdots,P_{n+1}=P'$, $n\leq d$, for each pair of $P_i,P_{i+1}$, assume $P_i$ receives a message at $t_i$, sends a message to $P_{i+1}$ at $t_i'$, and $P_{i+1}$ receives message at $t_{i+1}$, we can find $t_i'-t_i\leq\tau$, $t_{i+1}-t_i'\leq\nu$. Then we have
  	\[C_{n+1}(t)\geq C_{n+1}^{t_{n+1}}(t) \geq C_1(t_1)+(1-\kappa)(t-t_1)-n\xi\]
  \end{enumerate}
  \end{block}
}

\frame
{
  \frametitle{Physical Clock}
  \begin{block}{Step 2}
  \begin{enumerate}
  	\item<1-> Assign a clock $C_m$ to each message $m$ sent at $t$ and received at $t'$, with $C_m(t)=t$ and constant rate $dC_m/dt=\mu_m/(t'-t)$
  	\item<2-> For any time $t_x\geq t_0+\mu/(1-\kappa)$, let $C_x$ be the clock with largest value at $t_x$. Consider two cases:
  		\begin{enumerate}
  			\item<3-> $C_x$ is $C_q$ for some process $P_q$, then for any $i$ and $t\geq t_x$
  			\[C_i(t)\leq C_q(t_x)+(1+\kappa)(t-t_x)\]
  			\item<4-> $C_x$ is $C_m$ for some message $m$ sent at $t_1$ by process $P_q$, then $t_x\leq t_1+\mu/(1-\kappa)$, and for any $i$ and $t\geq t_1$
  			\[C_i(t)\leq C_q(t_1)+(1+\kappa)(t-t_1)\]
  		\end{enumerate}
  	\item<5-> Let $t_1=t_x$ in first case, then we have $t_1$ such that $t_x-\mu/(1-\kappa)\leq t_1\leq t_x$, and there exists process $P_q$ such that for all $t\geq t_x$ and for all $i$ the above equation holds.
  \end{enumerate}
  \end{block}
}

\frame
{
  \frametitle{Physical Clock}
  \begin{block}{Step 3}
  \begin{enumerate}
  	\item<1-> Now we conclude that there always exists process $P_q$ and time $t_1$ such that
  	\[C_q(t_1)+(1-\kappa)(t-t_1)-d\xi\leq C_i(t)\leq C_q(t_1)+(1+\kappa)(t-t_1)\]
  	\item<2-> Let $t=t_x+d(\tau+\nu)$, update the above bounds
  	\begin{eqnarray*}C_q(t_1)+(t-t_1)-\kappa d(\tau+\nu)-d\xi\leq C_i(t)\\\leq C_q(t_1)+(t-t_1)+\kappa[d(\tau+\nu)+\mu/(1-\kappa)]\end{eqnarray*}
  	\item<3-> By $\mu\leq\nu\ll\tau$ and $\kappa\ll 1$, and the fact that the above holds for all $i$
  	\[|C_i(t)-C_j(t)|\lesssim d(2\kappa\tau+\xi)\]
  \end{enumerate}
  \end{block}
}

\frame
{
  \center\huge{Q\&A}
}

\end{document}
